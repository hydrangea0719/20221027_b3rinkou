%\documentstyle[epsf,twocolumn]{jarticle}       %LaTeX2.09仕様
\documentclass[twocolumn]{jarticle}     %pLaTeX2e仕様
%%%%%%%%%%%%%%%%%%%%%%%%%%%%%%%%%%%%%%%%%%%%%%%%%%%%%%%%%%%%%%
%%
%%  基本 バージョン
%%
%%%%%%%%%%%%%%%%%%%%%%%%%%%%%%%%%%%%%%%%%%%%%%%%%%%%%%%%%%%%%%%%
\setlength{\topmargin}{-45pt}
%\setlength{\oddsidemargin}{0cm} 
\setlength{\oddsidemargin}{-7.5mm}
%\setlength{\evensidemargin}{0cm} 
\setlength{\textheight}{24.1cm}
%setlength{\textheight}{25cm} 
\setlength{\textwidth}{17.4cm}
%\setlength{\textwidth}{172mm} 
\setlength{\columnsep}{11mm}

\kanjiskip=.07zw plus.5pt minus.5pt


%【節がかわるごとに(1.1)(1.2) …(2.1)(2.2)と数式番号をつけるとき】
%\makeatletter
%\renewcommand{\theequation}{%
%\thesection.\arabic{equation}} %\@addtoreset{equation}{section}
%\makeatother

%\renewcommand{\arraystretch}{0.95} 行間の設定

%%%%%%%%%%%%%%%%%%%%%%%%%%%%%%%%%%%%%%%%%%%%%%%%%%%%%%%%
\usepackage[dvipdfm]{graphicx}   %pLaTeX2e仕様(要\documentstyle ->\documentclass)
%%%%%%%%%%%%%%%%%%%%%%%%%%%%%%%%%%%%%%%%%%%%%%%%%%%%%%%%

\begin{document}

\twocolumn[
\noindent
\hspace{1em}

輪講資料  2022 年 ?? 月 ?? 日(?)
\hfill
\ \ 一研太郎

\vspace{2mm}
\hrule
\begin{center}
{\Large \bf B3 輪講 LaTex 課題}
\end{center}
\hrule
\vspace{3mm}
]

\section{はじめに}
進捗報告や研究発表会の際の資料などに望ましい LaTeX の構成例です.\par
この PDF の例を見本として,「practice.tex」を元に同じものを作成してください.\par
名前や日付は適切なものに書き換えてください.

\subsection{課題での留意点}

\begin{itemize}
\item 〜practice〜 の部分を適宜補う
\item 図表を適切な位置に張り付ける
\item 図表と式に対しては \textbackslash label と \textbackslash ref を使う
\item bibtex を使って参考文献をのせる
\end{itemize}

\section{要素技術}

\subsection{要素技術 1}
〜practice〜 

\subsection{要素技術 2}
〜practice〜 

\section{提案手法}
提案手法はうんたらかんたら〜

\section{実験}
実験をかくかくしかじか〜

\section{実験結果}
〜practice〜 

\section{今後の課題}
今後の課題はうんぬんかんぬん〜

\bibliography{index}
\bibliographystyle{junsrt} %参考文献出力スタイル

\end{document}
